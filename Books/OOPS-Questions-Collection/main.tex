
\documentclass[14pt]{extarticle}
% Set target color model to RGB
\usepackage[inner=2.0cm,outer=2.0cm,top=2.5cm,bottom=2.5cm]{geometry}
\usepackage{setspace}
\usepackage[rgb]{xcolor}
\usepackage{verbatim}
\usepackage{amsgen,amsmath,amstext,amsbsy,amsopn,tikz,amssymb,tkz-linknodes}
\usepackage{fancyhdr}
\usepackage[colorlinks=true, urlcolor=blue,  linkcolor=blue, citecolor=blue]{hyperref}
\usepackage[colorinlistoftodos]{todonotes}
\usepackage{rotating}
%\usetikzlibrary{through,backgrounds}
\hypersetup{%
pdfauthor={Om Shri Prasath},%
pdftitle={OOPS Questions Collection},%
pdfcreator={PDFLaTeX},%
pdfproducer={PDFLaTeX},%
}
%\usetikzlibrary{shadows}
% \usepackage[francais]{babel}
\usepackage{booktabs}
\usepackage{etoolbox}
\usepackage[pygopt={texcomments=true,style=emacs}]{pythontex}


\newcommand{\ra}[1]{\renewcommand{\arraystretch}{#1}}

\newtheorem{thm}{Theorem}[section]
\newtheorem{prop}[thm]{Proposition}
\newtheorem{lem}[thm]{Lemma}
\newtheorem{cor}[thm]{Corollary}
\newtheorem{defn}[thm]{Definition}
\newtheorem{rem}[thm]{Remark}
\numberwithin{equation}{section}

\newcommand{\homework}[6]{
   \pagestyle{myheadings}
   \thispagestyle{plain}
   \newpage
   \setcounter{page}{1}
   \noindent
   \begin{center}
   \framebox{
      \vbox{\vspace{2mm}
      \vspace{6mm}
       \begin{center}
        { { {\Large \hfill #1  \hfill} } }
       \end{center}
       \begin{center}
        { { { \rm \it #3} } }
       \end{center} 
       %\hbox to 6.28in { {\it TA: #4  \hfill #6}}
      \vspace{2mm}}
   }
   \end{center}
   \markboth{#5 -- #1}{#5 -- #1}
   \vspace*{4mm}
}

\newcommand{\problem}[2]{~\\\fbox{\textbf{Question #1. #2}}\hfill \newline}
\newcommand{\subproblem}[1]{~\newline\textbf{(#1)}}
\newcommand{\D}{\mathcal{D}}
\newcommand{\Hy}{\mathcal{H}}
\newcommand{\VS}{\textrm{VS}}
\newcommand{\solution}[1]{\noindent \textbf{\textit{(A)}}{#1} \vspace{6mm}}

\newcommand{\bbF}{\mathbb{F}}
\newcommand{\bbX}{\mathbb{X}}
\newcommand{\bI}{\mathbf{I}}
\newcommand{\bX}{\mathbf{X}}
\newcommand{\bY}{\mathbf{Y}}
\newcommand{\bepsilon}{\boldsymbol{\epsilon}}
\newcommand{\balpha}{\boldsymbol{\alpha}}
\newcommand{\bbeta}{\boldsymbol{\beta}}
\newcommand{\0}{\mathbf{0}}

% \definecolor{listinggray}{gray}{0.9}
% \definecolor{lbcolor}{rgb}{0.9,0.9,0.9}
% \lstset{
% backgroundcolor=\color{lbcolor},
%     tabsize=4,    
% %   rulecolor=,
%     language=[GNU]C++,
%         basicstyle=\scriptsize,
%         upquote=true,
%         aboveskip={1.5\baselineskip},
%         columns=fixed,
%         showstringspaces=false,
%         extendedchars=false,
%         breaklines=true,
%         prebreak = \raisebox{0ex}[0ex][0ex]{\ensuremath{\hookleftarrow}},
%         frame=single,
%         numbers=left,
%         showtabs=false,
%         showspaces=false,
%         showstringspaces=false,
%         identifierstyle=\ttfamily,
%         keywordstyle=\color[rgb]{0,0,1},
%         commentstyle=\color[rgb]{0.026,0.112,0.095},
%         stringstyle=\color[rgb]{0.627,0.126,0.941},
%         numberstyle=\color[rgb]{0.205, 0.142, 0.73},
% %        \lstdefinestyle{C++}{language=C++,style=numbers}’.
% }
% \lstset{
%     backgroundcolor=\color{lbcolor},
%     tabsize=4,
%   language=C++,
%   captionpos=b,
%   tabsize=3,
%   frame=lines,
%   numbers=left,
%   numberstyle=\tiny,
%   numbersep=5pt,
%   breaklines=true,
%   showstringspaces=false,
%   basicstyle=\footnotesize,
% %  identifierstyle=\color{magenta},
%   keywordstyle=\color[rgb]{0,0,1},
%   commentstyle=\color{Darkgreen},
%   stringstyle=\color{red}
%   }

\begin{document}
\homework{OOPS Questions Collection}{}{Om Shri Prasath}{}{Om Shri Prasath}{NetId(s)}


\problem{1}{What is OOPS ?}

\solution{
\textbf{O}bject \textbf{O}riented \textbf{P}rogramming 
\textbf{S}tructure \textbf{(OOPS)} is a programming 
paradigm that organizes software design around 
\emph{objects}. The software is structured like 
different objects communicating among each other. 
An object is a collection of \emph{data} and 
\emph{methods} that operate on that data.
}

\problem{2}{What are all the advantages of OOPS?}

\solution{
    The advantages provided by OOPS are :
    \begin{itemize}
        \item \textbf{Simplicity :} The software objects are modelled after real-world objects, thus the structure of the program is easy to understand.
        \item \textbf{Modularity :} Each object in the program forms a separate entity which is decoupled from other objects, thus reducing complexity and chances of error from outside the object.
        \item \textbf{Modifiability :} Each object can be modified with minimal changes due to the modular nature of the code, thus changing one object is easy and also would not majorly affect the program as a whole.
        \item \textbf{Extensibility :} New features can be added to the program my adding or modifying existing objects. 
        \item \textbf{Maintainability :} Objects are maintained separately, making finding and fixing errors easier
        \item \textbf{Re-usability :} The objects can be reused easily in different programs due to its modular nature.
    \end{itemize}
}
\problem{3}{What are all the main features of OOPS?}

\solution{
    The main feature of \textbf{OOPS} are :
    
    \begin{itemize}
        \item \emph{Polymorphism}
        \item \emph{Encapsulation}
        \item \emph{Inheritance}
        \item \emph{Abstraction}
    \end{itemize}

    \noindent These are the key features implemented in different OOPS languages.
}


\problem{4}{What is Polymorphism?}

\solution{
    \textit{Polymorphism} refers to ability of the object to take different forms in different situations, or when used differently.
    \newline \newline
    A real-life example would be a man acts as a \textit{husband/father} in home, a \textit{worker} in office, and a \textit{consumer} in market. 
\newline\newline
    \noindent C++ code example :
    }
    \vspace{-2em}

    \begin{listing}[h!]
        \begin{pygments}[]{c++}
        # include <iostream>
        
        int main()
        {
           std::cout << "Hello, world!\n"; //\codeline{Important code line!}
        }
        \end{pygments}
    \end{listing}
        
    

\end{document} 




